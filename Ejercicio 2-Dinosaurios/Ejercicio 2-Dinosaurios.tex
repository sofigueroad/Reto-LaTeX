
%% %-----Preambulo ----- % % %

%Tipo de Documento: 
\documentclass{article} 

%Este paquete es utilizado para personalizar elementos por idioma: 
\usepackage[spanish ,mexico]{babel} 
  
%Este paquete es utilizado para la codificaci\’on de entrada: 
\usepackage[utf8]{inputenc} 

%Este paquete se utiliza para separar el texto en columnas
\usepackage{multicol}
\setlength{\columnseprule}{1pt}

%Otras declaraciones: 
\title{Art\'iculo: Dinosaurios...} 
\author{Steven  Spieldberg} 
\date{09 de julio de 2018} 

%%%----Cuerpo----- %%%
\begin{document}  
\maketitle 
\section{Introducci\'on}
Los dinosaurios \footnote{Dinosauria, del griego deinos sauros,``lagarto terrible''} son un grupo de reptiles \footnote{saur\'opsidos} que aparecieron durante el per\'iodo Tri\'asico \footnote{hace unos 231 millones de a\~nos}. Fueron los vertebrados terrestres dominantes durante 135 millones de a\~nos, desde el inicio del Jur\'asico \footnote{hace unos 200 millones de a\~nos} hasta el final del Cret\'acico \footnote{hace 66 millones de a\~nos}, cuando la mayor\'ia de los grupos de dinosaurios se extingui\'o durante el evento de estinci\'on del Cret\'acico - Pale\'ogeno que Puso fin a la Era Mezosoica.\\
\indent
Tendremos al final de este documento (en la p\'agina 2) la tabla ilustrativa de taxonom\'ia 2.1.

\section{Taxonom\'ia}
La siguiente es una clasificaci\'on simplificada de los grupos de dinosaurios, en funci\'on de sus relaciones evolutivas, y organizados bas\'andose en la lista de especies de dinosaurios mesozoicos facilitados por Holtz:\\

\begin{enumerate}
\item Saurischia
	\begin{itemize}	
	\item[UNO]Theropoda
		\begin{itemize}
		\item[un]Herrerasauria
		\item[deux]Coelophysoidea
		\item[trois]Tetanurae
			\begin{itemize}
			\item Megalosauroidea
			\item Carnosauria
			\end{itemize}
		\end{itemize}
	\item[DOS]Sauropodomorpha
	\end{itemize}
\end{enumerate}
En la subsecci\'on 2.1 observaremos las principales caracter\'isticas anat\'omicas distintivas.

\newpage
\subsection{Caracter\'isticas anat\'omicas distintivas}
\begin{tabular}{ l  r  l}
\hline
Anatom\'ia: & Dinosaurios\\
\hline \hline
Algunas & Los exocciptiales & No se juntan.\\\cline{2-2}
caracter\'isticas & Cresta deltopectoral & Situado al nivel o m\'as de 30\% .\\ \cline{2-3}
de los & En el cr\'aneo & Se presenta una fosa supratemporal.\\ \cline{2-2} 
reptiles & Epip\'ofisis & Procesos oblicuos.\\ \cline{2-3}
m\'as & Radio & Un hueso del brazo inferior, m\'as corto que el 80\% .\\ \cline{2-2}
antiguos & En la pelvis & Separados por gran superficie c\'oncava.\\ \cline{2-3}
conocidos & Astr\'agalo, calc\'aneo & Huesos del tobillo superiores.\\
\hline \bigskip \bigskip
\end{tabular}

Una evaluaci\'on detallada de las interrelaciones entre arcosaurios de S. Nesbitt confirmaron o encontraron las sinapomorfias inequ\'ivocas de la tabla 2.1.

\section{El reciente renacimiento de los dinosaurios}
El campo de onvestigaci\'on de los dinosaurios ha disfrutado de una oleada en la actividad que comenz\'o en los a\~nos 1970 y sigue en curso Desde el punto de vista de los seres humanos, los dinosaurios son criaturas que llaman la atenci\'on porque la mayor\'ia fueron de gran tama\~no y ten\'ian aspecto fant\'astico. Por este motivo han cautivado la imaginaci\'on de la gente y se han hecho famosos en la cultura popular desde finales del siglo XIX.

\end{document} 