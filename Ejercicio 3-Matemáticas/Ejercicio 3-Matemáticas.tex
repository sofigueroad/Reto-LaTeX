
%% %-----Preambulo ----- % % %

%Tipo de Documento: 
\documentclass{report} 

%Este paquete es utilizado para personalizar elementos por idioma: 
\usepackage[spanish ,mexico]{babel} 
  
%Este paquete es utilizado para la codificaci\’on de entrada: 
\usepackage[utf8]{inputenc} 

%Este paquete se utiliza para separar el texto en columnas
\usepackage{multicol}
\setlength{\columnseprule}{1pt}

%Otras declaraciones: 
\title{Un poco de Matem\'aticas}
\date{}



%%%----Cuerpo----- %%%
\begin{document}  
\maketitle 

\section{Serie de Fourier}

{\Huge U}{\large na serie de \textbf{\emph{Fourier}}\textit{es una serie infinita} que converge puntualmente a una funci\'on peri\'odica y continua a trozos (o por partes).}

\bigskip
\bigskip

\begin{flushright}
{\tiny Las} {\scriptsize series}{ \footnotesize de} {\small Fourier} {\normalsize constituyen} {\large la} {\Large herramienta} {\LARGE matem\'atica}\\
{\huge b\'asica} {\Huge del} {\huge an\'alisis} {\LARGE de} {\Large Fourier} {\large empleado para}\\
\large analizar funciones peri\'odicas a trav\'es de la descomposici\'on de\\
dicha funci\'on en  una suma infinita de funciones sinusoidales mucho\\
m\'as simples (como combinaci\'on de senos y cosenos con frecuencias\\
enteras). \texttt{El nombre se debe al matem\'atico franc\'es}\\
\texttt{Jean-Baptiste Joseph Fourier, que desarroll\'o la teor\'ia}\\
\texttt{cuando estudiaba la ecuaci\'on del calor.}
\end{flushright}

\smallskip

\textsc{\large Las series de} \textbf{\LARGE Fourier} \textsc{\large tienen la forma:}

\smallskip

\begin{eqnarray}
\frac{a_0}{2}\sum_{i=1}^{\infty}[a_n \cos\frac{2n\pi}{T}t + b_n \sin\frac{2n\pi}{T}]
\end{eqnarray}
{\large Donde $a_n$ y $b_n$ se denominan coeficiente de Fourier de la serie de Fourier de la funci\'on $f(x)$ en 1.}

\end{document}